\section{secci\'on}
\subsection{subsecci\'on}
  \begin{enumerate}
    \item Algo
    \item Algo
  \end{enumerate}

\subsection{secci\'on}
  \begin{lstlisting}[language=Python]
    import socket

    UDP_IP = "127.0.0.1"
    UDP_PORT = 5005
    MESSAGE = "Hello, World!"

    print "UDP target IP:", UDP_IP
    print "UDP target port:", UDP_PORT
    print "message:", MESSAGE

    sock = socket.socket(socket.AF_INET, # Internet
                          socket.SOCK_DGRAM) # UDP
    sock.sendto(MESSAGE, (UDP_IP, UDP_PORT))
  \end{lstlisting}

\subsection{subsecci\'on}
  %Eliminar la sangría al comienzo de un párrafo
  \noindent
  Esta es una cita \cite{einstein}\\
  Esta es otra cita \cite{dirac}\\
  Esta es otra cita \cite{knuthwebsite}\\

\subsection{subsecci\'on}
  \noindent
  \url{www.google.com.ar}

\subsection{subsecci\'on}
\begin{center}
    \begin{tabular}{ | l | l | l | p{5cm} |}
    \hline
    Day & Min Temp & Max Temp & Summary \\ \hline
    Monday & 11C & 22C & A clear day with lots of sunshine.
    However, the strong breeze will bring down the temperatures. \\ \hline
    Tuesday & 9C & 19C & Cloudy with rain, across many northern regions. Clear spells
    across most of Scotland and Northern Ireland,
    but rain reaching the far northwest. \\ \hline
    Wednesday & 10C & 21C & Rain will still linger for the morning.
    Conditions will improve by early afternoon and continue
    throughout the evening. \\
    \hline
    \end{tabular}
\end{center}

\begin{center}
\begin{tabular}{|c|c|c|}
  \hline
  \textbf{asd}     & \textbf{asd} & \textbf{asd}\\\hline\hline
      &			     			  & 123 \\ \cline{3-3}
  dsa& \multirow{-2}{*}{as} & 123\\ \hline
      &			                  & 123 \\ \cline{3-3}
  dsa& \multirow{-2}{*}{as} & 123\\ \hline
              &			& 123\\ \cline{3-3}
                & \multirow{-2}{*}{as} & 123\\ \cline{2-2}\cline{3-3}
               & 			&123\\ \cline{3-3}
  \multirow{-4}{*}{dsa} & \multirow{-2}{*}{as} & 123\\\hline
\end{tabular}
\end{center}

\subsection{subsecci\'on}
\href{https://www.overleaf.com}{Overleaf}\\\\\\
\href{http://www.google.com.ar}{\includegraphics[width=5cm]{google.png}}\\\\\\
\href{run:Documents/prueba.pdf}{Archivo}

\subsection{subsecci\'on}

\begin{figure}[H]
\centering
\includegraphics{prueba.pdf}
\end{figure}

\begin{mybox}{titulo}
  Algo
\end{mybox}
