%Define el tipo de documento, tamaño de letra y tipo de papel
\documentclass[12pt, a4paper]{article}
%Incluye el archivo style.sty
\usepackage{Preamble/style}
%Setea la ruta de las imágenes. Se pueden agregar más directorios de igual forma entre las llaves exteriores
\graphicspath{{Images/}}
%Setea la ruta de las referencias
\addbibresource{References/references.bib}
\newtcolorbox{mybox}[1]
{
	colframe = blue!25!,
	colback  = blue!10!,
	title    = #1
}
%Soluciona el problema de no poder poner ñ en los comentarios de códigos
\lstset{inputencoding=utf8/latin1}
\lstset{literate=
         {á}{{\'a}}1
         {í}{{\'i}}1
         {é}{{\'e}}1
         {ó}{{\'o}}1
         {ú}{{\'u}}1
         {ü}{{\"u}}1
      	 {ñ}{{\~n}}1 
         {Á}{{\'A}}1
         {Í}{{\'I}}1
         {É}{{\'E}}1
         {Ó}{{\'O}}1
         {Ú}{{\'U}}1
         {Ü}{{\"U}}1
}

%Personalizar encabezado y pie de página
\pagestyle{fancy}
\fancyhf{}
\lhead{Algo}
\rhead{Algo}
\rfoot{P\'agina \thepage}
\begin{document}
  \begin{titlepage}
  \vspace*{10cm}
  \begin{center}
      \Huge\textbf{\LaTeX}
  \end{center}
\end{titlepage}

  \newpage
  \begin{titlepage}
    \tableofcontents
  \end{titlepage}
  \include{Sections/section}
  %Imprime las referencias. Se puede poner en cualquier parte del documento. Para no imprimir el encabezado colocar [heading=none]
  %En la sección donde se desea que se impriman las referencias si agregamos "\nocite{*}" (sin comillas) agrega las referencias sin necesidad de citarlas
  \nocite{*}
  \printbibliography
  %Agrega las referencias a la tabla de contenidos
  \addcontentsline{toc}{section}{Referencias}
\end{document}
